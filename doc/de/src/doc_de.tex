\documentclass[12pt, a4paper, titlepage, hidelinks]{scrreprt}	
\input{src/preamble}

\hypersetup{
 pdfauthor={Mathias Garbe},
 pdftitle={Labor Algorithmen Dokumentation},
 pdfsubject={},
 pdfkeywords={}
}

\title{Labor Algorithmen - Sortierverfahren}
\subtitle{Dokumentation}
\author{Mathias Garbe (47777)}

\begin{document}

\pagenumbering{roman}
\maketitle

\microtypesetup{protrusion=false}
\tableofcontents
\microtypesetup{protrusion=true}

\clearpage

\pagenumbering{arabic}

\chapter{Sortieralgorithmen}

\section{Insertion Sort}

Der Insertion Sort Algorithmus ist ein leicht zu implementierendes und stabiles Sortierverfahren. Es nimmt aus der unsortierten Folge ein Element und fügt es vorne an richtiger Stelle der Folge wieder ein (daher auch \textit{Insertion Sort}), wobei die übrigen Elemente der Folge beim Einfügen nach hinten verschoben werden müssen. Dies wird nun an jeder Stelle der Folge von Anfang bis Ende durchgeführt bis die Folge sortiert ist. Der große Teil des Aufwands beim Insertion Sort liegt hierbei im Finden der richtigen Einfügeposition und das Verschieben der übrigen Elemente.

\paragraph{Variation: Wächterelement} Bei dieser Variation wird das kleinste Element vor dem Sortieren mit dem Insertion Sort Algorithmus an die erste Stelle der Folge verschoben. Dieses Element wird auch \textit{Wächterelement} genannt. Somit kann im Rumpf des Algorithmus die Abfrage auf ein Unterschreiten der Folge eingespart werden, was gegebenenfalls die Laufzeit reduzieren kann.

\paragraph{Variation: Wächterelement und Indextransformation} Zusätzlich zum Wächterelement wird hierbei die Berechnung des neuen Indexes an den Anfang der Schleife verschoben und somit kann gegebenenfalls die Laufzeit reduziert werden, da der neue Index nur einmal anstatt mehrmals berechnet werden muss.

\clearpage

\begin{SCfigure}
  \centering
  \includegraphics[width=0.5\textwidth]{graphs/Insertion-Sort.png}
  \caption*{\textbf{Schritt 1}: Das erste Element wird als sortiert angenommen und übersprungen. \\[45pt] %
  	\textbf{Schritt 2}: Das zweite Element wird überprüft und ist an der richtigen Stelle der Folge. \\[40pt] %
  	\textbf{Schritt 3}: Die \textit{2} ist nicht an der richtigen Stelle und muss nach an der zweite Stelle eingefügt werden. Alle nachfolgenden Elemente müssen verschoben werden. \\[15pt] %
  	\textbf{Schritt 4}: Das vierte Element wiederum muss nicht verschoben werden. \\[42pt] %
  	\textbf{Schritt 5}: Hier muss nun die \textit{6} vor die \textit{7} an der vierten Stelle eingefügt werden und alle nachfolgenden Elemente verschoben werden. \\[28pt] %
  	\textbf{Schritt 6}: Auch hier wird muss \textit{3} weit am Anfang eingefügt und viele nachfolgende Elemente verschoben werden.\\[32pt] %
  	\textbf{Schritt 7}: Die \textit{5} wird an der fünften Stelle eingefügt und alle nachfolgenden Elemente verschoben. \\[26pt] %
  	\textbf{Schritt 8}: Das letzte Element muss nun an der zweiten Position eingefügt werden, und beinahe die komplette Folge muss verschoben werden. \\[16pt] %
  	\textbf{Schritt 9}: Die Folge ist nun vollständig sortiert.
  }
\end{SCfigure}
\clearpage
\section{Quicksort}
Der Quicksort Algorithmus (vom englischen \textit{quick}, also `schnell') ist ein rekursives Sortierverfahren. Es ist im Gegensatz zu Mergesort und Insertion Sort nicht stabil. Es arbeitet nach dem ``teilen und herrschen''-Prinzip (\textit{divide and conquer}). Das Grundprinzip von Quicksort besteht im Aufteilen einer Folge anhand eines Pivotelements. In unserem Fall ist das Pivotelement immer das mittlere Element. Nun werden alle Elemente die kleiner als das Pivotelements auf die linke Seite verschoben und alle Elemente die größer sind rechts davon. Dabei werden diese Elemente nicht (wie etwa bei Insertion Sort) sortiert eingefügt, sondern nur verschoben. Danach wird die Folge in zwei Teilfolgen aufgeteilt und der Prozess für beide Teilfolgen wiederholt. Nachdem alle Teilfolgen nur noch ein Element haben ist die komplette Folge sortiert.

Im Durchschnitt kommt Quicksort mit $\mathcal{O}(n\log{}n)$ Vergleichen aus, im schlimmsten Fall $\mathcal{O}(n^2)$. In der Praxis ist Quicksort meist schneller als andere $\mathcal{O}(n\log{}n)$ Algorithmen. Auch sein sequentieller Speicherzugriff zusammen mit seiner sehr kurzen Schleife ist prädestiniert um von dem Prozessorcache zu profitieren.

\paragraph{Beispiel} 
\imagewrap{0.45}{graphs/quicksort.png}{Quicksort Beispiel}
\textbf{Schritt 1:} Als Pivot wird der Mittelpunkt der Folge genommen, hier die \textit{4}. Alle Werte kleiner als des Pivots werden auf die linke Seite verschoben, alle Werte größer des Pivots auf die rechte Seite. Danach wird die Folge in der Mitte partitioniert.

\textbf{Schritt 2 und Schritt 3:} Für alle Partitionen wird Schritt 1 wiederholt bis jede Partition nur noch aus einem Element besteht.

\textbf{Schritt 4:} Jede Partition besteht nur noch aus einem Element. Die Folge ist demnach vollständig sortiert.

\paragraph{Variation: Shiftoperator statt Division} Bei dieser Variante wird die Mitte der Partition nicht durch eine Division durch 2 ermittelt sondern durch einen Rechtsshift um 1.

\paragraph{Variation: 3-way-partitioning mit Shiftoperator} Bei dem 3-way-partitioning-Ansatz wird Quicksort um eine dritte Partition erweitert. So werden alle Elemente die den selben Wert wie das Pivotelement haben in die Mitte verschoben. Sollten keine Elemente einen gleichen Wert haben ist der Algorithmus nicht langsamer als das normale Quicksort.

\paragraph{Variation: Hybrider Ansatz} Bei dem hybriden Ansatz soll der 3-way-partitioning-Algorithmus genommen werden und um Aspekte von Insertion Sort und Mergesort erweitert werden. Wenn eine Partition kleiner als ein bestimmter Schwellwert ist so wird für die gesamte Partition Insertion Sort benutzt und keinerlei Rekursion mehr verwendet. Sollte dies jedoch nicht der Fall sein, so wird der 3-way-partitioning-Algorithmus benutzt. Vor dem rekursivem Aufruf wird jedoch überprüft, ob die Anzahl der Elemente der linken und rechten Partitionen ähnlich sind. Sollte der Unterschied zu groß sein so werden die Partitionen jeweils zwar mit einem rekursiven Aufruf sortiert, jedoch danach per Mergesort-Merge zusammengeführt. Die Rekursion vertieft in diesem Fall nicht weiter.

\clearpage

\section{Mergesort}
Das Mergesort Sortierungsverfahren arbeitet nach dem Prinzip des \textit{teilen und herrschen} (\textit{divide and conquer}). Auch dieser Algorithmus sortiert Folgen stabil. Es betrachtet die unsortierte Folge als eine Liste von Teilfolgen, welche am Anfang nur ein Element besitzen. Diese Teilfolgen werden dann im Reißverschlussverfahren zu größeren Folgen zusammengefügt (das \textit{mergen}). Dabei kann Mergesort davon ausgehen, dass jede Teilfolge sortiert ist. Dies ist im ersten Schritt sichergestellt, da jede Teilfolge nur aus ein Element besteht und somit schon sortiert ist. Um nun zwei sortierte Teilfolgen zusammenzufügen müssen diese Teilfolgen jeweils von links nach rechts durchgegangen werden und jeweils der kleinere Wert aus der Teilfolge genommen und in die neue Folge hinten angefügt werden. Dies wird solange wiederholt, bis beide Teilfolgen leer sind. Somit entsteht eine neue, sortierte Folge. Wird dies nun für alle Teilfolgen wiederholt bis es nur noch eine Folge gibt so ist diese gesamte Folge sortiert.

Die Arbeit des Aufteilen einer unsortierten Folge ist trivial, das eigentliche Zusammenfügen ist bei Mergesort jedoch die eigentliche Arbeit. Auch kann Mergesort nur sehr aufwändig ohne einen temporären Speicher ausgeführt werden (\textit{in-place}) und ist somit in der Regel auf den Speicherplatz einer zweiten, gleichgroßen Folge angewiesen.

Mergesort benötigt sowohl im durchschnitt als auch im besten und schlechtesten Fall stets $\mathcal{O}(n\log{}n)$ Vergleiche.

\paragraph{Beispiel}
\imagewrap{0.45}{graphs/mergesort.png}{Mergesort}
\textbf{Schritt 1:} Die unsortierte Folge wird als Teilfolge mit jeweils einem Element angesehen.

\textbf{Schritt 2:} Jeweils zwei Teilfolgen werden zu einer neuen Teilfolge zusammengefügt. Dabei werden die einzelnen Elemente sortiert nacheinander an die neue Folge angehangen.

\textbf{Schritt 3:} Da jede vorhergehende Teilfolge sortiert ist, kann auch hier wieder einfach die Elemente der vorhergehenden Teilfolgen sortiert an die neue Folge angehangen werden.

\textbf{Schritt 4:} Es existiert nur noch eine Folge und der Algorithmus terminiert. Die Folge ist somit sortiert.

\paragraph{Variation: Natural Mergesort} Bei dem Natural Mergesort wird zuerst versucht natürlich vorkommende Teilstücke in der eigentlich unsortierten Anfangsfolge zu finden, die bereits sortiert sind. Durch das Ausnutzen dieser Teilsortierung lassen sich gegebenenfalls Schritte des Mergesorts einsparen. Es können dabei sowohl aufsteigend als auch absteigend sortierte Teilfolgen benutzt werden (\textit{Bitonische Läufe}).

\chapter{Zeitmessungen}
Die Zeiten der einzelnen Messungen sind Durchschnittswerte von fünf unabhängigen Läufen mit jeweils unterschiedlichen Zufallsfolgen. Dabei wird pro Lauf jeder Algorithmus mit den gleichen Zufallsfolgen getestet. Die auf- und absteigenden Folgen unterscheiden sich nicht zwischen den Läufen.

Alle Zeiten sind in Millisekunden angegeben.

\clearpage
\pagestyle{empty}
\begin{landscape}
\begin{table}[h]
\begin{tabular}{|l|l|l|l|l|l|l|l|}
\hline
 & 10000 & 20000 & 40000 & 80000 & 160000 & 320000 & 640000 \\ \hline
InsertionSort & 13 & 52 & 207 & 857 & 3496 & 17417 & 57109 \\ \hline
InsertionSortGuard & 19 & 77 & 313 & 1039 & 4152 & 21865 & 72085 \\ \hline
InsertionSortGuardTransformed & 14 & 56 & 228 & 949 & 3839 & 19893 & 65926 \\ \hline
MergeSortBottomUp & 0 & 1 & 3 & 7 & 14 & 37 & 69 \\ \hline
MergeSortNatural & 0 & 2 & 3 & 7 & 16 & 39 & 75 \\ \hline
MergeSortTopDown & 4 & 7 & 15 & 31 & 63 & 163 & 271 \\ \hline
QuickSort & 0 & 1 & 2 & 5 & 10 & 24 & 47 \\ \hline
QuickSortShift & 0 & 1 & 2 & 5 & 10 & 25 & 48 \\ \hline
ShellSort & 175 & 744 & 3090 & 12655 & 50029 & 230128 & 826412 \\ \hline
\end{tabular}
\caption{Random}
\end{table}
\begin{table}[h]
\begin{tabular}{|l|l|l|l|l|l|l|l|}
\hline
 & 10000 & 20000 & 40000 & 80000 & 160000 & 320000 & 640000 \\ \hline
InsertionSort & 0 & 0 & 0 & 0 & 0 & 0 & 1 \\ \hline
InsertionSortGuard & 0 & 0 & 0 & 0 & 0 & 0 & 1 \\ \hline
InsertionSortGuardTransformed & 0 & 0 & 0 & 0 & 0 & 0 & 1 \\ \hline
MergeSortBottomUp & 0 & 0 & 1 & 3 & 6 & 15 & 29 \\ \hline
MergeSortNatural & 0 & 0 & 0 & 0 & 0 & 1 & 2 \\ \hline
MergeSortTopDown & 3 & 6 & 13 & 27 & 54 & 139 & 229 \\ \hline
QuickSort & 0 & 0 & 0 & 0 & 1 & 4 & 7 \\ \hline
QuickSortShift & 0 & 0 & 0 & 1 & 1 & 4 & 7 \\ \hline
ShellSort & 0 & 0 & 0 & 0 & 0 & 0 & 0 \\ \hline
\end{tabular}
\caption{Ascending}
\end{table}
\begin{table}[h]
\begin{tabular}{|l|l|l|l|l|l|l|l|}
\hline
 & 10000 & 20000 & 40000 & 80000 & 160000 & 320000 & 640000 \\ \hline
InsertionSort & 26 & 103 & 421 & 1742 & 7041 & 35906 & 118667 \\ \hline
InsertionSortGuard & 39 & 153 & 618 & 2073 & 8308 & 43788 & 143994 \\ \hline
InsertionSortGuardTransformed & 29 & 113 & 456 & 1924 & 7786 & 40625 & 136397 \\ \hline
MergeSortBottomUp & 0 & 0 & 1 & 3 & 6 & 18 & 33 \\ \hline
MergeSortNatural & 0 & 1 & 1 & 4 & 9 & 22 & 41 \\ \hline
MergeSortTopDown & 3 & 6 & 13 & 27 & 54 & 139 & 229 \\ \hline
QuickSort & 0 & 0 & 0 & 0 & 1 & 4 & 7 \\ \hline
QuickSortShift & 0 & 0 & 0 & 0 & 1 & 4 & 7 \\ \hline
ShellSort & 136 & 547 & 2194 & 8957 & 36098 & 155439 & 566640 \\ \hline
\end{tabular}
\caption{Descending}
\end{table}

\end{landscape}

\end{document}
